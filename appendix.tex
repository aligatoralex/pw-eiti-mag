
\part*{Załączniki}
\appendix

\SetSectionFormatting[breakbefore]{chapter}
	{6em}
	{\FormatChapterHeading{0pt}{\Large Załącznik }{\LARGE }}
	{8em plus1em minus1em}

\chapter{Przykładowy tytuł załącznika}
\label{app:zal1}

Przykładowy załącznik.

\begin{verbatim}
 jakies polecenie
\end{verbatim}

\begin{lstlisting}[basicstyle=\footnotesize\tt, numbers=left, numberstyle=\tiny,
numberfirstline=true, captionpos=t, caption=Plik \texttt{plik1.txt}]
# Listing

listing

# listing!
\end{lstlisting}

\chapter{Przykładowy, kolejny załącznik}
\label{app:zal_x}


W wyniku pracy programu otrzymano:

\begin{lstlisting}[basicstyle=\scriptsize\tt, numbers=left, numberstyle=\tiny,
tabsize=2, numberfirstline=true, captionpos=t, caption=Plik \texttt{wynik.txt}]
Hello world!
\end{lstlisting}

\chapter{Zawartość płyty dołączonej do pracy}
\label{app:cd}

\begin{large}
\noindent Do pracy dołączono płytę CD zawierającą:
\begin{itemize}
	\renewcommand{\labelitemi}{$\bullet$}
	\item Treść pracy w formacie PDF.
	\item Źródła programu.
	\item Pliki wymienione w załącznikach~\ref{app:zal1} i~\ref{app:zal_x}.
\end{itemize}
\end{large}
