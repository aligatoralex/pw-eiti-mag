% arara: lualatex: { shell: yes }
% arara: biber
% arara: makeglossaries
% arara: lualatex: { shell: yes }
% arara: lualatex: { shell: yes }

% mgr.tex - root top


\documentclass{eitidypl}

\usepackage[polish]{babel}
\usepackage[T1]{fontenc}
\usepackage{graphicx}
\usepackage[utf8]{inputenc}
\usepackage{multirow}
\usepackage{setspace}
\usepackage{url}
\usepackage{listings}
\usepackage{times}
\usepackage[autostyle]{csquotes}
\usepackage{bookmark}
\usepackage[
    backend=biber,
    style=authoryear-icomp,
    sortlocale=de_DE,
    natbib=true,
    url=false,
    doi=true,
    eprint=false
]{biblatex}
\addbibresource{biblatex-examples.bib}

\usepackage[]{hyperref}
\hypersetup{
    colorlinks=true,
}

%%%%%%%%%%%%%%% xindy ??????
\usepackage[xindy,acronym,toc]{glossaries}
   \makeglossaries

\newacronym[longplural={Frames per Second}]{LB}{LB}{Load-Balancing}
\newacronym[longplural={Frames per Second}]{HA}{HA}{High-Availbility}
\newacronym[longplural={Frames per Second}]{fpsLabel}{FPS}{Frame per Second}

\newglossaryentry{Linux}
{
  name=Linux,
  description={is a generic term referring to the family of Unix-like
               computer operating systems that use the Linux kernel},
  plural=Linuces
}


%%%%%%%%%%%%%%%%%%%%%%%%%%%%%%%%%%%%%%%%%%%%%%%%%%%%%%%%%%%%%%%%%%%%%%%%%%%%%
%NOTES
% http://tex.stackexchange.com/questions/9796/how-to-add-todo-notes

\usepackage{xargs}                      % Use more than one optional parameter in a new commands
\usepackage[pdflatex,dvipsnames]{xcolor}  % Coloured text etc.
% 
\usepackage[colorinlistoftodos,prependcaption,textsize=tiny]{todonotes}
\newcommandx{\unsure}[2][1=]{\todo[linecolor=red,backgroundcolor=red!25,bordercolor=red,#1]{#2}}
\newcommandx{\change}[2][1=]{\todo[linecolor=blue,backgroundcolor=blue!25,bordercolor=blue,#1]{#2}}
\newcommandx{\info}[2][1=]{\todo[linecolor=OliveGreen,backgroundcolor=OliveGreen!25,bordercolor=OliveGreen,#1]{#2}}
\newcommandx{\improvement}[2][1=]{\todo[linecolor=Plum,backgroundcolor=Plum!25,bordercolor=Plum,#1]{#2}}
\newcommandx{\thiswillnotshow}[2][1=]{\todo[disable,#1]{#2}}



%\todo[inline]{The original todo note withouth changed colours.\newline Here's another line.}
%\unsure{Is this correct?}\unsure{I'm unsure about also!}
%\change{Change this!}
%\info{This can help me in chapter seven!}
%\improvement{This really needs to be improved!\newline\newline What was I thinking?!}

%\%thiswillnotshow{This is hidden since option `disable' is chosen!}
%\improvement[inline]{The following section needs to be rewritten!}
%\lipsum[11]
%\newpage
%\listoftodos[Notes]


% %%%%%%%%%%%%%%%%%%%%%%%%%%%%%%%%%%%%%%%%%%%%%%%%%%%%%%%%%%%%%%%%%%%%%%%%%%%%




\rokegz{2017 r.}
\rokak{2017/2017}
\semestr{Lato}
\stopien{II}
\kierunek{Telekomunikacja}
\instytut{telekomunikacji}
\typ{magisterska}
\specjalnosc{ ??? Inżynieria Systemów Informatycznych}
\autor{Mikołaj Kowalski}
\adresa{ul. Puławska 5/12}
\adresb{02-111 Warszawa}
\foto{foto}
\dataurodzenia{31 grudnia 1990r.}
\datarozpoczecia{1. stycznia 2000r.}
\opiekun{dr hab. inż. Wojciech Mazurczyk}
\tytul{Ataki odmowy usługi oraz sposoby im przeciwdziałania w sieciach operatorskich}
\tytulen{Denial of Service in telecommunication networks -- attacks and mitigation}

\zyciorys{Urodziłem się 1 stycznia 1981r. w Warszawie...\\[3mm]
abc...\\[3mm]
def...\\[3mm]
Poniżej podpis:}

\streszczenie{Polskie streszczenie pracy...\\[3mm]
Dalsza część streszczenia...\\[3mm]
I coś jeszcze}

\streszczenieen{English abstract...\\[3mm]
Something more...\\[3mm]
And something else...}

\slowakluczowe{polskie, słowa, kluczowe, pracy}
\slowakluczoween{english, keywords}

%  %%%%%%%%%%%%%%%%%%%%%%%%%%%%%%%%%%%%%%%%%%%%%%%%%%%%%%%%%%%%%%%%%%%%%%%%%%%%

\begin{document}

% Polecenie \makeinfo produkuje karte informacyjna studenta.
% Sklada sie ja w dziekanacie razem z praca, ale NIE jest
% ona czescia pracy i nie powinna sie w niej znajdowac!
% Po wydrukowanie jednej karty informacyjnej mozna to polecenie
% usunac i drukowac tylko sama prace.
%\makeinfo

\maketitle

\makebio

\makeabstracts

\begin{titlepage}
	\noindent
	\vspace{15cm}
	\flushright
	\begin{minipage}{11cm}
			Serdecznie dziękuję XyZ.\\
			
			Pragnę także ...\\
		\end{minipage}
\end{titlepage}

\tableofcontents


\chapter{Wstęp: znaczenie niezawodnej infrastruktury sieciowej}

Wstęp do pracy.

\section{Zastosowanie testów sieci i urządzeń sieciowych}
\subsection{Dlaczego warto testować infrastrukturę}

Sprawdzenie możliwości architektury\\
Symulacja ataków (pentesty)\\
Poznanie realnej wydajności infrastruktury


\subsection{Dlaczego warto testować urządzenia sieciowe}

Zgodność ze specyfikacją \\
Szukanie podatności w urządzeniach \\
Rola testów przy zakupach (nowych inwestycjach ) - spełnienie wymagań projektowych


\section{Wprowadzenie do wysokiej dostępności (ang. \gls{HA}) i równoważenia obciążenia (ang. \gls{LB})}


\info{(ten rozdział jest ponieważ praca dotyczy również testowania architektury, w testach przewidziany jest load-balancing, zob. schematy labu)}

Co to jest wysoka dostępność i dlaczego to robimy, SPoF

Cel uzyskania niezawodnej i optymalnie wykorzystanej architektury


\subsection{Algorytmy \gls{LB}}

\subsection{Znaczenie session-persistence}

\subsection{Przykłady}

\begin{description}
\item[Alteon VADC] asd
\item[HAProxy] asd
\item[Keepalived + pacemaker] asd
\end{description}
\chapter{Przegląd generatorów ruchu sieciowego}
\section{Generatory sprzętowe}

charakterystyka, przykłady
\begin{itemize}
\item Spirent
\item Ixia
\end{itemize}





\section{Generatory programowe w Linuksie}

\subsection{Metody generowania wielkowolumenowego ruchu}

Opis procesu generowania pojedynczego pakietu w Linuksie \\
Po co robić memory zero-copy \\
Szybkie vs wolne backendy: SOCKET\_RAW, libpcap, netmap, PF\_RING, AF\_PACKET \\

\subsection{Funkcjonalności różnych generatorów/frameworków}
Badanie: netsniff-ng, scapy, PKTGEN)

\subsection{Wyspecjalizowane generatory}
 JMeter
\subsection{Fuzzery}

Na tą chwilę brak wiedzy/doświadczenia z tego typu programami




\section{Generatory – analiza komparatywna}

 (tabelka zalety-wady)
 
	Funkcjonalność vs Wydajność vs Cena
	
\begin{center}
  \begin{tabular}{ | l || c | c | c |}
    \hline
    Metoda & Funkcjonalność & Wydajność & Cena \\
    \hline \hline
    A & 15 & 15 & 1\\
    \hline
    B & 10 & 15 & 2 \\
    \hline
    C & 12 & 13 & 3\\
    \hline
    D & 110 & 230 & 4\\
    \hline
  \end{tabular}
\end{center}

\section{Metody analizy ruchu sieciowego}
\improvement{lepsza klasyfikacja}
\subsection{Metody opierające się na (kopii) ruchu}
\subsubsection{Urządzenie in-line}
\paragraph{Kopia ruchu (port-mirroring)}
\paragraph{Backendy: netmap, PF\_RING, pcap}
\subsubsection{Metody statystyczne}
\paragraph{Flowy: sFlow, NetFlow}
\paragraph{SNMP/Netconf}
\subsection{Analiza komparatywna (tabelka)}

\begin{center}
  \begin{tabular}{ | l || c | c | }
    \hline
    \multirow{2}{*}{Algorytm} & \multicolumn{2}{|c|}{Czas symulacji [sek]} \\
     & implementacji X & implementacji Y \\
    \hline \hline
    A & 15 & 15 \\
    \hline
    B & 10 & 15 \\
    \hline
    C & 12 & 13 \\
    \hline
    D & 110 & 230 \\
    \hline
  \end{tabular}
\end{center}


\chapter{Wprowadzenie do ataków odmowy usługi i istniejące sposoby przeciwdziałania}

 \info{w tym rodziale ma się znaleźć także przegląd literatury naukowej związanej z DDoS i obroną i na tym tle pokazanie o czym będzie Pana praca}
 
\section{Motywy/powody ataków}
\subsection{Straty wizerunkowe, odpływ klientów, okupy, kasa dla botmasterów}
\section{Możliwe skutki ataków}
\subsection{Kilka przykładów historycznych medialnych ataków}
\section{Charakterystyka ataków za rok 2016 w sieci OPL}
\begin{verbatim}
4.3.1.	Średnie natężenie
4.3.2.	Szczytowy ruch
4.3.3.	Średnia długość trwania
4.3.4.	Szczytowa długość trwania
4.3.5.	Procentowo protokoły
4.3.6.	Atakujący wg kraju
4.3.7.	Inne – zobaczymy co się da wyciągnąć (więcej niż raport certu)
4.3.8.	Być może porównanie do 2015 i wyznaczenie trendu  \info{tak, porównanie to dobry kierunek}
\end{verbatim}



\section{Klasyfikacja ataków}
Nie mogę opisać wszystkich ataków które są na świecie, trzeba znaleźć kryterium stopu – 
\info{tu oczywiście trzeba dobrać odpowiednie kryterium, żeby najlepiej to odpowiadało tym atakom, które będą przeprowadzane w części eksperymentalnej pracy dyplomowej.}
Na razie lista jest wstępna, pisana z pamięci. Trzeba pamietac o multivector attacks 
\begin{enumerate}
\item Wg źródła
	\begin{enumerate}
		\item Strumieniowe - DoS
		\item Rozproszone (Distributed) - DDoS
		\item Rozproszone (Distributed) - DDoS
		\item Odbite (Reflected) – DRDoS
		\item Wzmocnione (Amplified) – DRADoS
	\end{enumerate}
\item Wg warstwy protokołu
\begin{verbatim}
4.4.2.1.	L3:
4.4.2.1.1.	GRE
4.4.2.2.	L4:
4.4.2.2.1.	TCP flood flagi: SYN, ACK, SYN-ACK, PSH, FIN, FRAG
4.4.2.2.2.	UDP flood, UDP fragment
4.4.2.2.3.	ICMP flood
4.4.2.2.4.	TCP out of state
4.4.2.3.	L6
4.4.2.3.1.	THC-SSL-DoS (HTTPS renegotiation flood)
4.4.2.4.	L7
4.4.2.4.1.	HTTP
4.4.2.4.1.1.	Flood (GET/POST)
4.4.2.4.1.2.	Low and slow
4.4.2.4.2.	SNMP, DNS+DNSSEC, NTP
\end{verbatim}
\end{enumerate}

\section{Mitygacja ataków DDoS }
\improvement{tutaj też można wprowadzić jakąś klasyfikację}
\subsection{Metody}

Tryb in-line\\
BGP Flowspec\\
Mitgacja w cloudzie / scrubbing center
\subsection{Rozwiązania na rynku}

Radware DefensePro + DefenseFlow\\
Arbor\\
FastNetMon \\
...



\chapter{Podsumowanie}
\label{chapter:end}

Podsumowanie.

Drugi paragraf.

Odniesienie do \cite{some_thesis}.

W rozdziale~\ref{sec:sekcja_w} przedstawiono cośtam, a w~\ref{sec:sekcja_v} coś innego.

Na rysunku~\ref{etykieta_obrazka} umieszczono pingwina :)


%\bibliographystyle{alpha}
%\bibliographystyle{plain}
%\bibliography{mgr}
\printbibliography 


%logic
%\thispagestyle{empty} %plain ?
\newcommand{\listappendicesname}{Spis załączników}
\newlistof{appendices}{apc}{\listappendicesname}
\newcommand{\appendices}[1]{\addcontentsline{apc}{appendices}{#1}}
%\newcommand{\newappendix}[1]{\subsection*{#1}\appendices{#1}}
\newcommand{\newappendix}[1]{\section{#1}\appendices{#1}}
%\addcontentsline{toc}{chapter}{\listappendicesname}
\renewcommand{\cftapctitlefont}{\bfseries\LARGE}

%print list
\listofappendices
%schema for enumerating
\label{listof}
%\chapter*{\listappendicesname}



%add app
\appendix
\addtocontents{toc}{\protect\setcounter{tocdepth}{0}}

\chapter*{Załączniki}
\renewcommand{\thesection}{Załącznik \Alph{section}:}

\newappendix{ap1}
\label{ap:ap1}
asdasd


\end{document}

